\section{Introducción}
    En el presente trabajo, se detalla la resolución computacional de tres
    problemas algorítmicos, describiendo los modelos utilizados y la
    implementación de las soluciones propuestas. En los tres casos, tras
    demostrar la correctitud de los algoritmos utilizados, se procede a
    deducir una cota teórica para su complejidad temporal, que luego es
    corroborada en forma empírica a través de una serie de experimentos.
    También se verifican hipótesis acerca del comportamiento del algoritmo
    en situaciones de mejor y de peor caso, cuando resulta pertinente.

    Los dos primeros problemas son resueltos por medio de grafos, utilizando
    variantes de conocidos algoritmos para afrontar este tipo de escenarios.
    Para el tercero, por su parte, se utiliza la técnica de programación
    dinámica.

    En los tres casos, se programaron implementaciones de los algoritmos
    en el lenguaje C++, utilizando en varios casos funciones y
    estructuras de datos provistas por la biblioteca estándar del lenguaje.
    Estas implementaciones fueron el objeto de estudio de los experimentos
    mencionados, durante los cuales se tuvieron en cuenta las siguientes
    consideraciones generales:

    \begin{itemize}
    \item Solo se midió el costo temporal de generar las soluciones,
    omitiendo el demandado para la lectura y escritura de los datos.
    \item Para la medición de los tiempos de ejecución, se utilizaron los
    métodos proporcionados por el \emph{header} \texttt{chrono} de la
    biblioteca estándar de C++. Los tiempos fueron medidos en nanosegundos.
    \item Todas las pruebas se corrieron en las computadoras del Laboratorio X
    del Departamento de Computación (\acr{fcen} - \acr{uba}).
    \end{itemize}
