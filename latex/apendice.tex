\section{Apendice: Código fuente de las funciones principales}\label{sec:codigo}

\subsection{Ejercicio 1 (Una nueva esperanza)}
\begin{lstlisting}
stack<int> resolver(const vector<vector<pair<int,bool>>> &lista_ady,
                                                             int n){
  // Representacion para los estados de un nodo
  // 0..n - 1: No se recorrio ningun pasillo especial
  // n..2*n - 1: Se recorrio un pasillo especial
  // 2*n..3*n - 1: Se recorrieron al menos dos pasillos especiales
  vector<int> lista_cuevas = vector<int>(3*n, -1);

  queue<int> cola_nodos;

  // Al nodo incial lo marco como su propio padre
  lista_cuevas[0] = 0;
  cola_nodos.push(0);

  int id_nodo, id_nodo_efectivo, id_nodo_ady;
  // Mientras queden nodos sin visitar
  while(!cola_nodos.empty()){
    id_nodo = cola_nodos.front();
    id_nodo_efectivo = id_nodo % n;
    cola_nodos.pop();
    // Encolo todos los vecinos
    for(unsigned int i=0; i<lista_ady[id_nodo_efectivo].size(); i++){
      id_nodo_ady = (id_nodo - id_nodo_efectivo) 
        + lista_ady[id_nodo_efectivo][i].first;

      // Si tengo un camino especial, encolo el nodo del grafo 
      // siguiente
      if(lista_ady[id_nodo_efectivo][i].second == true && 
        id_nodo_ady/n < 2){
        id_nodo_ady += n;
      }

      // Si ya recorri este nodo no es necesario encolarlo
      if(lista_cuevas[id_nodo_ady] == -1){
        lista_cuevas[id_nodo_ady] = id_nodo;
        cola_nodos.push(id_nodo_ady);
      }
    }
  }

  stack<int> camino_optimo;

  int j = lista_cuevas[3*n - 1];
  while(lista_cuevas[j] != j){
    camino_optimo.push(j % n);
    j = lista_cuevas[j];
  }

  return camino_optimo;
}
\end{lstlisting}

\subsection{Ejercicio 2 (El Imperio contraataca)}
\begin{lstlisting}
int resolver(const vector<vector<pair<int,int>>>& grafo, 
                              vector<int>& predecesores) {

  uint N = grafo.size();

  // Crear una cola de prioridad que contiene a todos los nodos
  // La prioridad inicial es infinita (maximo int posible)
  cola_prioridad vertices(N);

  // Establecer en 0 la prioridad del primer vertice
  vertices.setear_prioridad(0, 0);

  // En este vector se registraran los nodos que ya no deben procesarse
  vector<bool> vertices_marcados(N, false);

  // Aqui se ira formando la solucion del problema
  predecesores = vector<int>(N, 0);

  // Aqui se ira calculando el peso total del arbol
  int peso_total = 0;

  // Iterar hasta que la cola este vacia
  while (! vertices.vacia()) {
    // Desencolar el vertice con menor prioridad (esto representa al
    // vertice unido al arbol generado hasta el momento por la arista
    // mas corta posible). Marcarlo como visitado
    peso_total += vertices.min_prioridad();
    int vert_actual = vertices.min_indice();
    vertices.desencolar();
    vertices_marcados[vert_actual] = true;

    // Iterar sobre los vecinos no marcados del nodo seleccionado
    for (uint j = 0; j < grafo[vert_actual].size(); j++) {
      int ind_vecino  = grafo[vert_actual][j].first;

      if (! vertices_marcados[ind_vecino]) {
        int long_arista = grafo[vert_actual][j].second;

        // Comparar la longitud de la arista recien examinada con
        // el valor almacenado como prioridad del vecino, y en caso
        // de que sea necesario, actualizar este valor
        if (long_arista < vertices.prioridad(ind_vecino)) {
          vertices.setear_prioridad(ind_vecino, long_arista);
          predecesores[ind_vecino] = vert_actual;
        }
      }
    }
  }

  return peso_total;
}
\end{lstlisting}

\subsection{Ejercicio 3 (El retorno del jedi)}
\begin{lstlisting}
pair<int,vector<char>> resolver(vector<vector<pair<int,char>>> &DP, 
                                            vector<vector<int>> &E,
                                                             int N,
                                                             int M,
                                                             int H) {
  // Instanciar el DP
  DP = vector<vector<pair<int,char>>>(N,
    vector<pair<int,char>>(M, make_pair(-1,'Z')));
  DP[0][0] = pair<int,char>(0,'I');
  int costoY = 0;
  int costoX = 0;
  for (int i = 1; i < N; ++i){
    costoY = DP[i - 1][0].first + calcularEnergiaGastada(0, i-1, 0, i, H, E);
    DP[i][0] = pair<int,char>(costoY,'X');
    
  }
  for (int j = 1; j < M; ++j){
    costoX = DP[0][j - 1].first + calcularEnergiaGastada(j-1, 0, j, 0, H, E);
    DP[0][j] =  pair<int,char>(costoX,'Y');
  }

  for (int i = 1; i < N; ++i){
    for (int j = 1; j < M; ++j){
      // si me moviera por el eje Y
      int costoY = DP[i-1][j].first
        + calcularEnergiaGastada(j, i-1,j, i, H, E);
      
      // si me moviera por el eje X
      int costoX = DP[i][j-1].first
        + calcularEnergiaGastada(j-1, i, j, i, H, E);
      
      if (costoX > costoY){
        DP[i][j] = pair<int,char>(costoY,'X');
      }else{
        DP[i][j] = pair<int,char>(costoX,'Y');
      }

    }
  }

  int i = N - 1;
  int j = M - 1;
  vector<char> camino = vector<char>(N + M - 2);
  while(i + j > 0){
    camino[i + j - 1] = DP[i][j].second;
    if (DP[i][j].second == 'Y'){
      j--;
    }else{
      i--;
    }
  }
  
  pair<int,vector<char>> result = make_pair(DP[N-1][M-1].first,camino);
  return result;
}

int calcularEnergiaGastada(vector<vector<int>> &E,
                                     int xDestino,
                                     int yDestino,
                                      int xOrigen,
                                      int yOrigen,
                                            int H){
  
  int delta = abs(E[yOrigen][xOrigen] - E[yDestino][xDestino]);
  if(H > delta){
    return  0 ;
  }else{
    return delta - H;
  }
}
\end{lstlisting}
