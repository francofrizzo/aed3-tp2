\section{Ejercicio 3: El retorno del jedi}

    % Describir detalladamente el problema a resolver dando ejemplos del mismo y sus soluciones.
    \subsection{Descripción del problema}

    % Explicar de forma clara, sencilla, estructurada y concisa, las ideas desarrolladas para la resolución del problema. Utilizar pseudocódigo y lenguaje coloquial (no código fuente). Justificar por qué el procedimiento resuelve efectivamente el problema.
    \subsection{Solución propuesta}

    % Deducir una cota de complejidad temporal del algoritmo propuesto (en función de los parámetros que se consideren correctos) y justificar por qué el algoritmo la cumple. Utilizar el modelo uniforme.
    \subsection{Complejidad teórica}

    Calcular el salto vertical u horizontal tiene un costo de $O(1)$. Calcular la primera fila tiene costo $O(M)$, ya que he de recorrer la primera fila y guardar en el DP, a la suma de la anterior más el salto horizontal. Calcular la primera la primera columna cuesta $O(N)$, ya que he de recorrer la primera columna y guardar en el DP,  la suma de la anterior más el salto vertical .

    Para llenar el resto del DP hemos de recorrer en orden creciente por columnas, como en cada iteración tomo el minimo entre el costo salto vertical más venir en vertical y el horizontal más venir en horizontal ($O(1)$) la complejidad para terminar de llenar el DP es de $O(N*M)$.

    Para Armar el camino se va recorriendo la matriz de la posición final, (fila $N - 1$ , columna $M - 1$) hasta la inicial (fila $0$ , columna $0$), en cada paso me fijo en la DP de donde vine, lo guardo y me muevo a ese ($O(1)$). Esto tiene un total de $N + M - 2$ iteraciones ya que en cada paso me muevo a la izquierda ó para arriba. Entoces armar el camino cuesta $O(N + M - 2)$

    El costo total es $O( M + N + M*N + (N + M − 2) ) = O(N*M)$
     
    \subsection{Experimentación}

    % Ejemplo de gráfico para reutilizar:
     \begin{figure}[H]
         \centering
         \caption{}
         \label{fig:exp1:tiempo_base}
         \begin{tikzpicture}
             \begin{axis}[
                     title={},
                     xlabel={Tamaño de entrada ($N$)},
                     ylabel={Tiempo de ejecución (nanosegundos)},
                     scaled x ticks=false,
                     scaled y ticks=false,
                     enlargelimits=0.05,
                     width=0.5\textwidth,
                     height=0.5\textwidth,
                     legend pos=south east,
                     legend cell align=left,
                     xmin=1
                 ]
                 \addplot[color=black] table[x index=0,y index=1]{../exp/elRetornoDelJediFilas};
                 \addplot[color=red] table[x index=0, y expr={x*ln(x)*10}]{../exp/elRetornoDelJediFilas};
                 \legend{$T$, $c*N*N$}
             \end{axis}
         \end{tikzpicture}
     \end{figure}

     \begin{figure}[H]
         \centering
         \caption{}
         \label{fig:exp1:tiempo_base}
         \begin{tikzpicture}
             \begin{axis}[
                     title={},
                     xlabel={Tamaño de entrada ($N$)},
                     ylabel={Tiempo de ejecución (nanosegundos)},
                     scaled x ticks=false,
                     scaled y ticks=false,
                     enlargelimits=0.05,
                     width=0.5\textwidth,
                     height=0.5\textwidth,
                     legend pos=south east,
                     legend cell align=left,
                     xmin=1
                 ]
                 \addplot[color=black] table[x index=0,y index=1]{../exp/elRetornoDelJediColumnas};
                 \addplot[color=red] table[x index=0, y expr={x*ln(x)*10}]{../exp/elRetornoDelJediColumnas};
                 \legend{$T$, $c*N*N$}
             \end{axis}
         \end{tikzpicture}
     \end{figure}

     \begin{figure}[H]
         \centering
         \caption{}
         \label{fig:exp1:tiempo_base}
         \begin{tikzpicture}
             \begin{axis}[
                     title={},
                     xlabel={Tamaño de entrada ($N$)},
                     ylabel={Tiempo de ejecución (nanosegundos)},
                     scaled x ticks=false,
                     scaled y ticks=false,
                     enlargelimits=0.05,
                     width=0.5\textwidth,
                     height=0.5\textwidth,
                     legend pos=south east,
                     legend cell align=left,
                     xmin=1
                 ]
                 \addplot[color=black] table[x index=0,y index=1]{../exp/elRetornoDelJediFilasYColumnas};
                 \addplot[color=red] table[x index=0, y expr={x*ln(x)*10}]{../exp/elRetornoDelJediFilasYColumnas};
                 \legend{$T$, $c*N*N$}
             \end{axis}
         \end{tikzpicture}
     \end{figure}

     \begin{figure}[H]
         \centering
         \caption{}
         \label{fig:exp1:tiempo_base}
         \begin{tikzpicture}
             \begin{axis}[
                     title={},
                     xlabel={Tamaño de entrada ($N$)},
                     ylabel={Tiempo de ejecución (nanosegundos)},
                     scaled x ticks=false,
                     scaled y ticks=false,
                     enlargelimits=0.05,
                     width=0.5\textwidth,
                     height=0.5\textwidth,
                     legend pos=south east,
                     legend cell align=left,
                     xmin=1
                 ]
                 \addplot[color=black] table[x index=0,y index=1]{../exp/elRetornoDelJediFilas};
                 \addplot[color=red] table[x index=0, y expr={x*ln(x)*10}]{../exp/elRetornoDelJediFilas};
                 \legend{$T$, $c*N*N$}
             \end{axis}
         \end{tikzpicture}
     \end{figure}
