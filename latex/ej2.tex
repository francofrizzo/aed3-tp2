\section{Ejercicio 2: El Imperio contraataca}

    % Describir detalladamente el problema a resolver dando ejemplos del mismo y sus soluciones.
    \subsection{Descripción del problema}

    \begin{figure}[ht]
        \begin{center}
            \includegraphics[width=10cm]{imagenes/el_imperio_contraataca.jpg}
            \caption{¡Prestame atención a mí, soy tu padre!}
        \end{center}
    \end{figure}

    Luke debe informarle a todos sus aliados para que estén al tanto del contraataque del Imperio. Parte del planeta Hoth, conocido como planeta 0, y quiere que la noticia llegue a todos los planetas.

    En cada planeta hay infinitos halcones milenarios. Los halcones milenarios consumen una determinada de combustible cada un millón de kilómetros.

    Una vez que un halcón milenario llega a un planeta, de éste pueden salir cualquier cantidad de halcones para seguir advirtiendo a los rebeldes de otros planetas.

    Los halcones milenarios no pueden viajar por cualquier lado, solo por determinadas rutas espaciales. Cada ruta conecta exactamente dos planetas, y recorrer cada ruta consume una determinada cantidad de combustible. Además, se sabe que existe al menos una forma de llegar desde cualquier planeta a cualquier otro a través de rutas espaciales.

    Se pide encontrar cuál es la mínima cantidad de combustible necesaria para entregar el mensaje en todos los planetas. Además, se debe indicar de qué planeta proviene la nave que entrega el mensaje, para cada planeta excepto el 0.

    Se pide que el algoritmo tenga una complejidad temporal $\ord(M$ log $M)$.

    \vspace{1.25em}

    \textbf{Formato de entrada}: La primera línea consta de un entero positivo \texttt{N}, que indica la cantidad de planetas, y un entero positivo M, que indica la cantidad de rutas espaciales. A continuación de esta línea siguen \texttt{M} líneas con enteros \texttt{Ai}, \texttt{Bi} y \texttt{Li}, siendo \texttt{Ai} y \texttt{Bi} los extremos de la ruta y \texttt{Li} la cantidad de litros que se gastan al recorrer esa ruta ($0 \leq \mathtt{Ai} \neq \mathtt{Bi} \leq N-1$). La entrada contará con el siguiente formato:

    \begin{verbatim}
    N M
    A0 B0 L0 
    A1 B1 L1 
    ...
    AM-1 BM-1 LM-1\end{verbatim}

    \vspace{.8em}

    \textbf{Formato de salida}: La primera línea debe contener la cantidad mínima de litros \texttt{L} necesarios para informar a toda la alianza sobre esta noticia, seguida de $N-1$ líneas que indican desde qué planeta se viaja para informar de la situación a cada planeta (el vecino inmediato desde el cual se viaja). El formato debe ser el siguiente:
    
    \begin{verbatim}
    L
    I1
    I2
    ...
    IN-1\end{verbatim}

    indicando que al planeta \texttt{i} se le informa de la situación con una nave que parte del planeta \texttt{Ii}.

    \vspace{.8em}

    A continuación se incluyen, a modo de ejemplo, una posible entrada y una
    salida correcta para la misma:

    \vspace{.5em}
    \begin{tabular}{l @{\hskip 4em} l}
    \textbf{Entrada} & \textbf{Salida} \\
    \texttt{4 4}     & \texttt{4}      \\
    \texttt{0 1 1}   & \texttt{0}      \\
    \texttt{1 2 2}   & \texttt{1}      \\
    \texttt{1 3 5}   & \texttt{2}      \\
    \texttt{2 3 1}   &                 \\
    \end{tabular}
    \vspace{.5em}

    Es importante observar que pueden haber más de una salida distinta válida.

    % Explicar de forma clara, sencilla, estructurada y concisa, las ideas
    % desarrolladas para la resolución del problema. Utilizar pseudocódigo y
    % lenguaje coloquial (no código fuente). Justificar por qué el
    % procedimiento resuelve efectivamente el problema.
    \subsection{Solución propuesta}

    El hecho de que los planetas se encontraran unidos mediante rutas, y que
    cada una de ellas conectara exactamente dos planetas, permitió encarar la
    resolución del problema modelando el sistema planetario mediante un
    grafo, donde cada planeta es un vértice y cada ruta es un eje.

    Para representar los distintos costos de utilizar una u otra ruta
    espacial, se consideró un grafo ponderado, utilizando como peso de cada
    eje la cantidad de combustible que cuesta recorrer la ruta
    correspondiente. Por otra parte, el hecho de que siempre exista forma de
    viajar entre dos planetas equivale a decir que el grafo es conexo.

    Sea $G$ el grafo cuyo conjunto de vértices son los planetas y su conjunto
    de aristas, las rutas espaciales. Notar que $G$ es un grafo conexo y
    sus aristas tienen pesos no negativos. Bajo este modelo, el problema que
    busca resolverse se traduce en hallar un subgrafo $T$ de $G$, de forma tal
    que:
    \begin{enumerate}
        \item Todos los nodos de $G$ sean nodos de $T$.
        \item Para algún nodo $v_0 \in V(G)$ (representando al planeta Hoth),
        exista en $T$ un camino entre $v_0$ y cualquier otro nodo. Dado que
        $T$ es un grafo no dirigido, esto equivale a pedir que $T$ sea conexo.
        \item La suma de los pesos de todos los ejes $T$ sea mínima con
        respecto a todos los subgrafos de $G$ que cumplen los dos puntos
        anteriores.
    \end{enumerate}


    \begin{lema}
    Sea $G$ un grafo con pesos no negativos, y $T \subseteq G$ un árbol
    generador mínimo de $G$. Entonces $T$ cumple las tres propiedades recién
    enunciadas.
    \end{lema}
    \begin{proof}
    Sabemos que $T$ cumple los dos primeros ítems, por definición de árbol
    generador mínimo.
    Supongamos ahora que no se cumple el tercer punto, es decir, que existe
    otro subgrafo de $G$, $H$, que también cumple las primeras dos propiedades
    y tal que el peso total de sus aristas es estrictamente menor que el de
    $T$.

    Por el segundo punto, sabemos que $H$ es conexo, por lo que tiene al menos
    un árbol generador mínimo $\tilde{H}$ que también cumple las primeras dos
    propiedades enunciadas. Por el primer punto, $H$ contiene todos los
    nodos de $G$, así que $\tilde{H}$ también es árbol generador de $G$.
    Dado que las aristas de $H$ tienen pesos no negativos, y todas las aristas
    de $\tilde{H}$ están en $H$, el peso total de las aristas de $\tilde{H}$
    es menor o igual que la de $H$ y, por transitividad, también que el de
    $T$. Es decir, $H$ es un árbol generador de $G$ cuyo peso total es menor
    que el de $T$, lo cual cual contradice el hecho de que que $T$ es un árbol
    generador mínimo de $G$.

    Esto prueba que no puede existir tal subgrafo $H$, es decir, que $T$
    cumple también con la tercer propiedad.
    \end{proof}

    % Deducir una cota de complejidad temporal del algoritmo propuesto (en
    % función de los parámetros que se consideren correctos) y justificar por
    % qué el algoritmo la cumple. Utilizar el modelo uniforme.
    \subsection{Complejidad teórica}

    % Realizar experimentación para medir la performance, usando un conjunto
    % de casos de test que permitan observar los tiempos de ejecución en
    % función de los parámetros de entrada, tanto para instancias aleatorias
    % (detallando cómo fueron generadas) como para instancias particulares
    % (peor/mejor caso, por ejemplo). Presentar en forma gráfica una
    % comparación entre los tiempos medidos y la complejidad teórica y extraer
    % conclusiones.
    \subsection{Experimentación}

    Se realizaron pruebas experimentales para verificar que el tiempo de ejecución del algoritmo cumpliera con la cota asintótica de $\theta(M * log(N))$, demostrada teóricamente. Se realizaron las siguientes pruebas:   

    \begin{itemize}
    \item Pruebas con instancias en donde se mantiene la cantidad de nodos fija y se hacen variar las aristas.
    \item Pruebas con instancias en donde se mantiene la cantidad de aristas fija y se hacen variar los nodos.
    \end{itemize}

    Se generaron 100 instancias de prueba en cada uno de estos escenarios. En cada caso, se ejecutaron 60 repeticiones del algoritmo, midiendo cada vez el tiempo de ejecución y tomando luego el promedio entre los resultados obtenidos.

    Debido a los picos en el tiempo de ejecución en las primeras corridas del programa, tendiendo los valores a estabilizarse con las sucesivas corridas, se decidió, para minimizar la varianza de los datos obtenidos, tratar a las primeras 20 repeticiones de cada instancia como \textit{outliers}, y promediar solo los valores de las últimas 40.

    Todas las instancias utilizadas para estas pruebas se generaron de manera aleatoria. 

    Los resultados obtenidos se exponen en las figuras \ref{fig:exp2:n_fijo} y \ref{fig:exp2:m_fijo}.
    
    \newcommand\constante{16}

    En la figura \ref{fig:exp2:n_fijo} se reflejan los resultados de mantener una cantidad $N$ de nodos fija, variando la cantidad de aristas $M$ de los grafos. El valor en el que se fijó $N$ fue de 400. Los valores de $M$ se encuentran entre 399 y 79800. Estos valores corresponden a la mínima y máxima cantidad de aristas que un grafo conexo con $N$ fijo puede tener, respectivamente. Notar que necesitamos que el grafo sea conexo para cumplir con los requisitos del enunciado del problema. La mínima cantidad de aristas de un grafo conexo de $N$ nodos es de $M = N - 1$, y la máxima es el grafo completo, $M = \frac{N(N - 1)}{2}$.

    La compljidad del algoritmo es de orden $\ord(M \times log(N))$, pero como en esta prueba $N$ es constante, la complejidad temporal debería ser lineal en la cantidad de aristas. Es por eso que se ilustra la cota teórica de $c \times M$ (el valor de $c$ utilizado es \constante). 

    \begin{figure}[H]
        \centering
        \caption{}
        \label{fig:exp2:n_fijo}
        \begin{tikzpicture}
            \begin{axis}[
                    title={Cantidad de nodos fija $N = 400$},
                    xlabel={Cantidad de aristas($M$)},
                    ylabel={Tiempo de ejecución (nanosegundos)},
                    scaled x ticks=false,
                    scaled y ticks=false,
                    enlargelimits=0.05,
                    width=0.5\textwidth,
                    height=0.5\textwidth,
                    legend pos=south east,
                    legend cell align=left,
                    xmin=1
                ]
                \addplot[color=black] table[x index=0,y index=1]{../exp/elImperioContraatacaNFijo};
                \addplot[color=red] table[x index=0, y expr={x*\constante}]{../exp/elImperioContraatacaNFijo};
                \legend{$T$, $c \times M$}
            \end{axis}
        \end{tikzpicture}
    \end{figure}

    \renewcommand\constante{10000000}

    En la figura \ref{fig:exp2:m_fijo} se reflejan los resultados de mantener una cantidad $M$ de aristas fija, variando la cantidad de nodos $N$ de los grafos. El valor en el que se fijó $M$ fue de 400000. Los valores de $N$ se encuentran entre 895 y 400001. Estos valores corresponden a la mínima y máxima cantidad de nodos que un grafo conexo con $M$ fijo puede tener, respectivamente. La mínima cantidad de nodos de un grafo conexo de $M$ aristas es el grafo completo, $N = \frac{1 + \sqrt{1 + 8M}}{2}$, y la máxima es de $N = M + 1$.

    Así como en el experimento anterior, la complejidad temporal del experimento varía de la del algoritmo, ya que $M$ se mantiene constante. Esta debería ser logarítmica en la cantidad de nodos. Es por eso que se ilustra la cota teórica de $c \times log(N)$ (el valor de $c$ utilizado es \constante). 
    
    \begin{figure}[H]
        \centering
        \caption{}
        \label{fig:exp2:m_fijo}
        \begin{tikzpicture}
            \begin{axis}[
                    title={Cantidad de nodos fija $M = 400000$},
                    xlabel={Cantidad de nodos($N$)},
                    ylabel={Tiempo de ejecución (nanosegundos)},
                    scaled x ticks=false,
                    scaled y ticks=false,
                    enlargelimits=0.05,
                    width=0.5\textwidth,
                    height=0.5\textwidth,
                    legend pos=south east,
                    legend cell align=left,
                    xmin=1
                ]
                \addplot[color=black] table[x index=0,y index=1]{../exp/elImperioContraatacaMFijo};
                \addplot[color=red] table[x index=0, y expr={ln(x)*\constante}]{../exp/elImperioContraatacaMFijo};
                \legend{$T$, $c*log(N)$}
            \end{axis}
        \end{tikzpicture}
    \end{figure}

    \renewcommand\constante{12}
    \begin{figure}[H]
        \centering
        \caption{}
        \label{fig:exp2:arboles}
        \begin{tikzpicture}
            \begin{axis}[
                    title={Árboles},
                    xlabel={Cantidad de aristas($M$)},
                    ylabel={Tiempo de ejecución (nanosegundos)},
                    scaled x ticks=false,
                    scaled y ticks=false,
                    enlargelimits=0.05,
                    width=0.5\textwidth,
                    height=0.5\textwidth,
                    legend pos=south east,
                    legend cell align=left,
                    xmin=1
                ]
                \addplot[color=black] table[x index=0,y index=1]{../exp/elImperioContraatacaArboles};
                \addplot[color=red] table[x index=0, y expr={x*ln(x + 1)*\constante}]{../exp/elImperioContraatacaArboles};
                \legend{$T$, $c*M*log(N)$}
            \end{axis}
        \end{tikzpicture}
    \end{figure}
