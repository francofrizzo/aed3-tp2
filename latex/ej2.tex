\section{Ejercicio 2: El Imperio contraataca}

    % Describir detalladamente el problema a resolver dando ejemplos del mismo y sus soluciones.
    \subsection{Descripción del problema}

    % Explicar de forma clara, sencilla, estructurada y concisa, las ideas desarrolladas para la resolución del problema. Utilizar pseudocódigo y lenguaje coloquial (no código fuente). Justificar por qué el procedimiento resuelve efectivamente el problema.
    \subsection{Solución propuesta}

    % Deducir una cota de complejidad temporal del algoritmo propuesto (en función de los parámetros que se consideren correctos) y justificar por qué el algoritmo la cumple. Utilizar el modelo uniforme.
    \subsection{Complejidad teórica}

    % Realizar experimentación para medir la performance, usando un conjunto de casos de test que permitan observar los tiempos de ejecución en función de los parámetros de entrada, tanto para instancias aleatorias (detallando cómo fueron generadas) como para instancias particulares (peor/mejor caso, por ejemplo). Presentar en forma gráfica una comparación entre los tiempos medidos y la complejidad teórica y extraer conclusiones.
    \subsection{Experimentación}

    % Ejemplo de gráfico para reutilizar:
    % \begin{figure}[H]
    %     \centering
    %     \caption{}
    %     \label{fig:exp1:tiempo_base}
    %     \begin{tikzpicture}
    %         \begin{axis}[
    %                 title={},
    %                 xlabel={Tamaño de entrada ($N$)},
    %                 ylabel={Tiempo de ejecución (nanosegundos)},
    %                 scaled x ticks=false,
    %                 scaled y ticks=false,
    %                 enlargelimits=0.05,
    %                 width=0.5\textwidth,
    %                 height=0.5\textwidth,
    %                 legend pos=south east,
    %                 legend cell align=left,
    %                 xmin=1
    %             ]
    %             \addplot[color=black] table[x index=0,y index=1]{../exp/kaioKenOutput};
    %             \addplot[color=red] table[x index=0, y expr={x*ln(x)*\constante}]{../exp/kaioKenOutput};
    %             \legend{$T$, $c*N*log(N)$}
    %         \end{axis}
    %     \end{tikzpicture}
    % \end{figure}
